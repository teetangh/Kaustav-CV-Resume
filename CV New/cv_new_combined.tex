%%%%%%%%%%%%%%%%%%%%%%%%%%%%%%%%%%%%%%%%%
% "ModernCV" CV and Cover Letter
% LaTeX Template
% Version 1.3 (29/10/16)
%
% This template has been downloaded from:
% http://www.LaTeXTemplates.com
%
% Original author:
% Xavier Danaux (xdanaux@gmail.com) with modifications by:
% Vel (vel@latextemplates.com)
%
% License:
% CC BY-NC-SA 3.0 (http://creativecommons.org/licenses/by-nc-sa/3.0/)
%
% Important note:
% This template requires the moderncv.cls and .sty files to be in the same 
% directory as this .tex file. These files provide the resume style and themes 
% used for structuring the document.
%
%%%%%%%%%%%%%%%%%%%%%%%%%%%%%%%%%%%%%%%%%

%----------------------------------------------------------------------------------------
%	PACKAGES AND OTHER DOCUMENT CONFIGURATIONS
%----------------------------------------------------------------------------------------

\documentclass[11pt,a4paper,sans]{moderncv} % Font sizes: 10, 11, or 12; paper sizes: a4paper, letterpaper, a5paper, legalpaper, executivepaper or landscape; font families: sans or roman

\moderncvstyle{classic} % CV theme - options include: 'casual' (default), 'classic', 'oldstyle' and 'banking'
\moderncvcolor{blue} % CV color - options include: 'blue' (default), 'orange', 'green', 'red', 'purple', 'grey' and 'black'

\usepackage{lipsum} % Used for inserting dummy 'Lorem ipsum' text into the template
\usepackage[scale=0.90]{geometry} % Reduce document margins
%\setlength{\hintscolumnwidth}{3cm} % Uncomment to change the width of the dates column
%\setlength{\makecvtitlenamewidth}{10cm} % For the 'classic' style, uncomment to adjust the width of the space allocated to your name

%----------------------------------------------------------------------------------------
%	NAME AND CONTACT INFORMATION SECTION
%----------------------------------------------------------------------------------------

\firstname{Kaustav} % Your first name
\familyname{Ghosh} % Your last name

% All information in this block is optional, comment out any lines you don't need
\title{Curriculum Vitae}
\address{Gurgaon}{Haryana-122018}
\mobile{(+91)-88004411954}
% \phone{(000) 111 1112}
% \fax{(000) 111 1113}
\email{teetangh@gmail.com}
% \homepage{staff.org.edu/~jsmith}{staff.org.edu/$\sim$jsmith} % The first argument is the url for the clickable link, the second argument is the url displayed in the template - this allows special characters to be displayed such as the tilde in this example
% \extrainfo{additional information}

% \homepage{github.com/teetangh}{GitHub}
\homepage{linkedin.com/in/kaustav-ghosh-1538651bb/}{LinkedIn}
\photo[70pt][0.4pt]{pictures/kaustav2} % The first bracket is the picture height, the second is the thickness of the frame around the picture (0pt for no frame)

% \quote{"A witty and playful quotation" - John Smith}

%----------------------------------------------------------------------------------------

\begin{document}
% \vspace*{-15mm}


% %----------------------------------------------------------------------------------------
% %	COVER LETTER
% %----------------------------------------------------------------------------------------

% % To remove the cover letter, comment out this entire block

% \clearpage

% \recipient{HR Department}{Corporation\\123 Pleasant Lane\\12345 City, State} % Letter recipient
% \date{\today} % Letter date
% \opening{Dear Sir or Madam,} % Opening greeting
% \closing{Sincerely yours,} % Closing phrase
% \enclosure[Attached]{curriculum vit\ae{}} % List of enclosed documents

% \makelettertitle % Print letter title

% \lipsum[1-2] % Dummy text
% \lipsum[4] % Dummy text

% \makeletterclosing % Print letter signature

% \newpage

%----------------------------------------------------------------------------------------
%	CURRICULUM VITAE
%----------------------------------------------------------------------------------------

\makecvtitle % Print the CV title


%----------------------------------------------------------------------------------------
%	INTERESTS
%----------------------------------------------------------------------------------------

% \section{Interests}
%  Artificial Intelligence,
%  Machine Learning,
%  Deep Learning,
%  Reinforcement Learning,
%  Computer Vision,
%  Robotics

 \section{Interests}
 Computer Science,
 Artificial Intelligence
 and 
 Robotics
 

%----------------------------------------------------------------------------------------
%	EDUCATION SECTION
%----------------------------------------------------------------------------------------

\section{Education}

% \cventry{2011--2012}{Masters of Commerce}{The University of California}{Berkeley}{\textit{GPA -- 8.0}}{First Class Honours}  % Arguments not required can be left empty
\cventry{2018--2022}{Bachelor of Technology in Computer Science \& Engineering}{\newline Manipal Institute of Technology}{Manipal}{}%{\textit{CGPA -- 8.3}}
{Specializing in Computational Intelligence}

% \section{Bachelor's Thesis}

% \cvitem{Title}{\emph{Money Is The Root Of All Evil -- Or Is It?}}
% \cvitem{Supervisors}{Professor James Smith \& Associate Professor Jane Smith}
% \cvitem{Description}{This thesis explored the idea that money has been the cause of untold anguish and suffering in the world. I found that it has, in fact, not.}

%----------------------------------------------------------------------------------------
%	WORK EXPERIENCE SECTION
%----------------------------------------------------------------------------------------

\section{Work Experience}

\subsection{Robotics Internships}

%------------------------------------------------

\cventry{Jul'20-Aug'20}
{ROS Engineer Intern}
{\textsc{Qbotics Labs}}
{India}
{}
{
\textbf{ CEO and Mentor :} \href{https://in.linkedin.com/in/lentinjoseph}{Lentin Joseph (author of 8 ROS books)}.
\textbf{ Project:} \href{https://github.com/teetangh/Qbotics-Labs-Internship-Differential-Drives}{Repository}.
\begin{itemize}
    \item{Constructed a Differential Drive with caster wheel from scratch using URDF and XACRO files and mounted the same with laser scanner , Inertial Measurement Unit and Velodyne Puck VLP-16 Lidar. }
    \item{Simulated the differential drive in \underline{Gazebo} and wrote ROS Subscriber script to get laser scan reading from sensor messages for obstacle range detection }
    \item{Interfaced the differential drive with \underline{Google Cartographer} with localisation and mapping of the robot using lua config files. }
    % \item{Currently working on motion planning(including obstacle avoidance and wallfolowing)}
    \item{Modelled a 4 wheeled drive and also an environment for experimentation of various controllers for the vehicle in \underline{Webots 3D Robot Simulator}}
    \item{Wrote individual C++ controllers for the teleoperation using keyboard,laser scanner,GPS,IMU \& Linear Actuator}
    \item{Wrote \underline{Markdown documentation} for the entirety of the Internship for beginners to understand concepts and replicate results}
\end{itemize}
}

%------------------------------------------------
\subsection{AI internships}

\cventry{Apr-Jun'20}
{Machine Learning Intern}
{\textsc{Microsoft Student Partners}}
{India}
{}
{
\href{https://github.com/teetangh/Kaustav-All-Certifications/blob/master/Artificial\%20Intelligence/MSP\%20ML\%20Internship/internship\%20certificate.pdf}{ \textbf{\emph{ML Certificate}}.}
\href{https://github.com/teetangh/Kaustav-All-Certifications/blob/master/Artificial\%20Intelligence/MSP\%20ML\%20Internship/Course\%20Completion\%20Certificate.pdf}{ \textbf{\emph{Python Foundations} Certificate.}}
\href{https://github.com/Microsoft-ML-Internship-Team/Major-Project-Submissions}{ \textbf{\emph{Team Repository}}.}
\begin{itemize}
    \item {Guided a team of 10 individuals to collaborate and accomplish a Regression task of price prediction of used cars}
    \item {Performed Feature Engineering to extract the most important attributes of the data-set using Uni-variate and Multi-variate Filtering techniques, Mutual Entropy Gain Filtering and also feature selection using RMSE Regression and ANOVA Test}
    \item {Performed basic Data wrangling and processing using Numpy and Pandas and visualized it using Matplotlib and Seaborn and finally built the machine learning model using an XGboost Regressor}
    \item {Also completed a Mini Project on extensive Data Visualization and Analysis using Mat-plotlib and Seaborn to gather useful insights of the data}
\end{itemize}
\href{https://github.com/teetangh/Microsoft-Machine-Learning-Internship/blob/master/MINOR\%20PROJECT/Microsoft_Minor_Project_v2.ipynb}{\textbf{\emph{Mini Project}}}
\href{https://github.com/Microsoft-ML-Internship-Team/Major-Project-Submissions/blob/master/KAUSTAV/02_Kaustav_feature_engineering_v4.ipynb}{\textbf{\emph{ Feature Engineering Notebook}}.}
\href{https://github.com/Microsoft-ML-Internship-Team/Major-Project-Submissions/blob/master/KAUSTAV/03_Kaustav_Buidling_the_model_v1.ipynb}{\textbf{\emph{ Model Notebook}}.}
\href{https://github.com/Microsoft-ML-Internship-Team/Major-Project-Submissions/blob/master/KAUSTAV/01_Kaustav_data_preprocess_EDA_v7.ipynb}{\textbf{\emph{ EDA Notebook}}}
}
%------------------------------------------------


\cventry{May'20}
{Data Analytics Intern}
{\textsc{Takenmind Technologies}}
{India}
{}
{
\begin{itemize}
    \item{Scripted a personal version of Numpy and Pandas Documentation }
    \item{Performed Exploratory Data Analyis techniques using Matplotlib and Seaborn }
    \item{Creating several box-plots,count-plots,heat-maps of several data-sets}
\end{itemize}
}

\cventry{Ongoing}
{Deep Learning with Masters in Computer Vision and NLP}
{\textsc{Ineuron Intelligence}}
{India}
{}
{
\begin{itemize}
    \item Currently learning CNNs and RNNs
\end{itemize}
}
\newpage

%----------------------------------------------------------------------------------------
%	ACHIEVEMENTS SECTION
%----------------------------------------------------------------------------------------
\section{Achievements}

\cvitem{2021}{Secured a GPA of \textbf{9.28} in 5th Semester}
\cvitem{2020}{Among \textbf{top 4 out of 370} students in Competitive Coding contest to secure internship at \newline \textbf{Samsung Research Institute}, Summer '21}
\cvitem{2020}{Secured a GPA of \textbf{9.14} in 4th Semester}

%----------------------------------------------------------------------------------------
%	RESEARCH PROJECTS SECTION
%----------------------------------------------------------------------------------------
\section{Research Projects}

\cvitem{Nov'20-Ongoing}{
    \textbf{Samsung PRISM}
    {Intelligent Ranking for Dynamic Restoration in Next Generation Wireless Networks}
}   

%----------------------------------------------------------------------------------------
%	ACADEMIC PROJECTS SECTION
%----------------------------------------------------------------------------------------
\section{Academic Projects}

\cvitem{Compiler Design}{
    \textbf{Front End of a Compiler}
    \begin{itemize}
        \item{Coded a \textbf{Lexical Analyser} that extracts tokens from a C source file and a \textbf{Symbol Table Generator} to store information of identifiers and functions.\href{https://github.com/teetangh/KaustavLABS3/blob/main/CD\%20LAB/LAB\%2004/lab04_symbol_table_lexical_analyser_complete.c}{[\textbf{Code}]}} 
        \item{Coded a \textbf{Recursive Decent Parser} that semantically parses the \textit{grammar for subset of C-Language} by analysing the tokens generated by the Lexical Analyser and reports syntactic \& semantic errors  \href{https://github.com/teetangh/KaustavLABS3/blob/main/CD\%20LAB/LAB\%200789/lab09_RDP_main.c}{[\textbf{Code}]}}
    \end{itemize}
}

\cvitem{ Algorithms \& Data Structures}{
    \textbf{Backtracking Algorithms}
    \begin{itemize}
        \item{Coded a \textbf{Crossword Solver} that takes a 10*10 grid and word list and outputs a grid with the words accurately filled into the slots.\href{https://github.com/teetangh/Kaustav-Competitive-Coding/blob/master/CodeZen/03\%20Algorithms\%20and\%20Competitive\%20Programming/11\%20Backtracking/prog004crosswordSolver.cpp}{[\textbf{Code}]}} 
        \item{Coded a \textbf{Sudoku Solver} that takes a partially filled 9*9 sudoku grid and outputs a solution so that every row, column and nine 3x3 subgrids contains exactly 1 instance of the digits from 1 to 9. .\href{https://github.com/teetangh/Kaustav-Competitive-Coding/blob/master/CodeChef/Public/Code\%20Marathon/sudokuSolver.cpp}{[\textbf{Code}]}} 
    \end{itemize}
}  

\cvitem{Machine Learning}{
    \textbf{Finland Labs in association with NSS IIT Roorkee - Covid-19 Data Analysis, Time Series Forecasting and Web Scraping}
    \newline
    \textbf{Link to Certificate :} \href{https://github.com/teetangh/Kaustav-All-Certifications/blob/master/Artificial\%20Intelligence/Finland\%20Labs\%20and\%20IITR/Covid\%2019\%20Analysis\%20-\%20AI\%20and\%20ML.pdf}{Certificate}.
    \textbf{ Link to Code :} \href{https://github.com/teetangh/FinlandLabs-IITR-COVID-19-Analysis}{Repository}.
    \begin{itemize}
        \item{Prepared a complete Data Analysis report on the World-wide COVID-19 attack statistics and used the Facebook's fbprophet Time-series Forecasting library to speculate the number of active corona victim cases in the upcoming days.} 
        \item{Also used a corona data-set of my country and the Python folium package for the binding of data to a map for choropleth visualizations. Further used Beautiful Soup and Requests HTTP library for Web Scraping of live corona stats. }
        \item{Implemented code snippets for the pre-processing of data \& data wrangling and visualized the data via several Matplotlib and Seaborn tools }
        \item{Created neural networks from scratch which facilitated in implementing a machine learning model to recognize the function of an XOR gate without explicitly being programmed.}
        \item{Trained a Deep Learning model with TF2 and Keras API for MNIST Handwritten digit Recognition}
    \end{itemize}
    }

\cvitem{ROS-Gazebo}{
    \textbf{Food Labs Robotics Startup Interview}
    \newline
    \textbf{ Models and Simulations :} \href{https://github.com/teetangh/Kaustav-ROS-Workspace}{Repository \& videos}.
    \textbf{  Final Project Report:} \href{https://github.com/teetangh/Kaustav-ROS-Workspace/blob/master/Resources/Final\%20Report.pdf}{Final Report}.
    \begin{itemize}
        \item{Designed, modelled, constructed and assembled a plethora of sensors and Robots across multiple software platforms like FreeCad, Blender, Gazebo and also fabricated a hotel from floor-plan using Gazebo World Editor}
        \item{Created an SDF model of the Velodyne HDL-32 sensor,improved the model's appearance and data output,added Mass/Inertia to the model,used FreeCad software to acquire Meshes, Blendr software to refine the metric system and Gazebo model editor to model the Velodyne Lidar structure.}
        \item{Implemented Hokuyo Fake Laser Scanner and Noisy Camera in Gazebo, tweaked the mean \& standard deviation of the Gaussian Noise Distribution in the scan \& image samples for higher fidelity outputs.}
        \item{Simulated the ROBOTIS waffle-pi or burger TurtleBot3 and constructed a vehicle in Gazebo Model editor and loaded it with a Depth Camera Sensor for surveillance}
    \end{itemize}
}
        

\cvitem{Solidworks}{
    \textbf{Analysis of Selective Compliance Assembly Robot Arm and Modelling of T3R Robot}
    \begin{itemize}
        \item{Computed Denavit-Hartenberg parameters for the SCARA robot and used it to formulate the Forward and Inverse Kinematics of the robot arm} 
        \item{Used Lagrange Euler Formulation to compute the torque/dynamics of the robot and further also planned an arbitrary trajectory for the manipulator} 
        \item{Using Solidworks modelled a T3R robot (1 twisting joint and 3 revolute joints) and as bonus task i am trying to interface the Soliworks model with Matlab Simscape}
    \end{itemize}
    \textbf{ SCARA :} \href{https://github.com/teetangh/Robotics-Projects/blob/master/SCARA_Robot_Analysis.pdf}{SCARA Analysis}.
    \textbf{ T3R :} \href{https://github.com/teetangh/Robotics-Projects/tree/master/T3R\%20Robot}{Solidworks Model}.
    \textbf{ T3R animation :} \href{https://github.com/teetangh/Robotics-Projects/blob/master/T3R\%20Robot/Resources/T3R\%20Animation.mp4}{mp4 video}.
}





    
% \newpage
%----------------------------------------------------------------------------------------
%	Positions of Responsibility SECTION
%----------------------------------------------------------------------------------------

\section{Positions of Responsibility}

\cvitem{Jan'20 - Present}{Local Committee Member of IOSD (International Organization of Software Developers)}

%----------------------------------------------------------------------------------------
%	TECHNICAL SECTION
%----------------------------------------------------------------------------------------

\section{Technical Section}

% \cvitem{Basic}{\textsc{java}, Adobe Illustrator}
% \cvitem{Intermediate}{\textsc{python}, \textsc{html}, \LaTeX, OpenOffice, Linux, Microsoft Windows}
% \cvitem{Advanced}{Computer Hardware and Support}

\cvitem{Software:}{Anaconda,AutoCAD,Matlab,Keil,Altera MaxPlus 2,VirtualBox,Vm Ware,Oracle SQL,VS Code}
\cvitem{Programming Languages: }{Fluent in C/C++,Familiar with Java and Python,Verilog,\LaTeX,Linux Shell Scripting,
fair acquaintance with ARM assembly programming\textit{(NXP LPC 1768)}}
\cvitem{Libraries and Frameworks:}{\textit{Python}-Numpy, Pandas, SciPy,Scikit-Learn,Matplotlib, Keras, Tensorflow \textit{C++}-Standard Template Library(STL) \textit{Java}-JavaFX GUI}    
\cvitem{Robotics}{ROS middleware, Gazebo, Ignition, MoveIt!, Point Cloud Library}
% \cvitem{Web Dev}{HTML, CSS, JavaScript and familiarity with MERN stack}
\cvitem{OS used}{\textit{Windows}-XP,Vista,7,10 \textit{Linux}-Ubuntu}



%----------------------------------------------------------------------------------------
%	COURSES SECTION
%----------------------------------------------------------------------------------------

\section{Courses Taken}
    \subsection{College Curriculum}
        \cvitem{}{   Engineering Mathematics, 
        Data Structures,
        Object Oriented Programming with Java,
        Digital System Design with Verilog,
        Computer Organization and Architecture,
        Database Systems,
        Theory of Computation,
        Embedded Systems,
        Algorithms,
        Operating Systems,
        Computer Networks,
        Compiler Design,
        Software Engineering,
        Robotics,
        Smart Sensors
        }
  
    \subsection{Off-Campus Academies and Online Courses}
    \cvitem{Coding Ninjas}{Intro to C++ programming with Data Structures and won Top Performer Certificate of Excellence}
    \cvlistdoubleitem{\href{https://github.com/teetangh/Kaustav-All-Certifications/blob/master/Programming/Coding\%20Ninjas/Coding\%20Ninjas\%20Cpp\%20Completion\%20Certificate.pdf}{Link to Completion Certificate }}{\href{https://github.com/teetangh/Kaustav-All-Certifications/blob/master/Programming/Coding\%20Ninjas/Coding\%20Ninjas\%20Cpp\%20Top\%20Performer\%20Certificate.pdf}{Link to Top Performer Certificate }}
    % \cvitem{*} {\href{https://github.com/teetangh/Kaustav-All-Certifications/blob/master/Programming/Coding\%20Ninjas/Coding\%20Ninjas\%20Cpp\%20Completion\%20Certificate.pdf}{Link to Completion Certificate }}
    % \cvitem{*} {\href{https://github.com/teetangh/Kaustav-All-Certifications/blob/master/Programming/Coding\%20Ninjas/Coding\%20Ninjas\%20Cpp\%20Top\%20Performer\%20Certificate.pdf}{Link to Top Performer Certificate }}
    \cvlistitem {\href{https://github.com/teetangh/Kaustav-Competitive-Coding}{Link to Cpp, Data structures and Algorithms Repository}}

%----------------------------------------------------------------------------------------
%	EARLY YEARS SECTION
%----------------------------------------------------------------------------------------

\section{Early Years}
\begin{itemize}
    \item Pulled an all nigther in 7th grade to construct a \textbf{LEGO Mindstorms EV2 
    Humanoid Robot} and programmed it using the NXT-G GUI interface that could walk,talk 
    and identify colors in its environment and showcased it in a Science exhibition.
    \item Won Gold medal for 200m sprint and 2 bronze medals fro 100m and 400m in 8th grade.
    \item Won \textbf{1st Prize} Trophy at state Level Abacus Competition in 5th grade
    \item Won \textbf{2nd Prize} Trophy at state Level Abacus Competition in 4th grade     
    \item Won multiple gold, silver and bronze medals at the \textbf{School Level}  Science 
    and Maths Olympiads conducted by \textbf{Science Olympiad Foundation}            
\end{itemize}


%----------------------------------------------------------------------------------------
%	HOBBIES SECTION
%----------------------------------------------------------------------------------------

\section{Hobbies}

\renewcommand{\listitemsymbol}{-~} % Changes the symbol used for lists

\cvlistdoubleitem{Running}{Anime}
\cvlistitem{Music}

%----------------------------------------------------------------------------------------

\end{document}