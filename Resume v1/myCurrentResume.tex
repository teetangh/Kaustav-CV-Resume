\documentclass[a4paper,12pt]{article}
\usepackage[utf8]{inputenc}
\usepackage{titlesec}
\usepackage[margin=0.7in]{geometry}
\usepackage{multicol}
\usepackage{tabularx}

\addtolength{\topmargin}{-0.05in}
\begin{document}
    

% \setlength{\columnseprule}{0.4pt}

\begin{multicols}{2}
\section*{\LARGE{Kaustav Ghosh}}
    \section*{Contact Details:}
        Phone: +91-8800441954
        \newline
        Email: \textit{teetangh@gmail.com}
        \newline
        GitHub: \textit{https://github.com/teetangh}
        \newline
        
    \section*{Academic Description}
        \textbf{\emph{Manipal Institute of Technology}}
        \newline
        \textit{2018 – 2022 (ongoing)}
        \newline
        \textit{B. Tech in Computer Science and Engineering}
        \newline
        \textit{CGPA: 7.93}
        \newline
        \textit{Number of completed semesters: 3}
        \newline
        \textbf{\emph{Amity International School}}
        \newline
        \textit{12th CBSE Percentage: 91\% (2015 - 2017)}
        \newline
        \textbf{\emph{Suncity World School}}
        \newline
        \textit{10th CBSE CGPA: 9.8 (2013 - 2015)}
    \end{multicols}

\section*{Interests}
Artificial Intelligence,
Machine Learning,
Deep Learning,
Reinforcement Learning,
Robot Operating System
    
%%%%%%%%%%%%%%%%%%%%%%%%%%%%%%%%%%%%%%%%%%%%%%%%%%%%%%%%%%%%%%%%%%%%%%%%%%%%%%
    % \section*{\LARGE{Kaustav Ghosh}}
    % Phone: +91-8800441954
    % \newline
    % Email: \textit{teetangh@gmail.com}
    % \newline
    % GitHub: \textit{https://github.com/teetangh}

    % % {\begin{tabularx}{\dimexpr\linewidth-25pt}[t]{@{}XXX@{}}
    % % \raggedright
    % % \section*{\Large{Kaustav Ghosh}}
    % %             Phone: +91-8800441954
    % %             \newline
    % %             Email: teetangh@gmail.com
    % %             \newline
    % %             GitHub: https://github.com/teetangh
        
    % %         & 
    % % % \centering  & 
    % % \raggedleft 
    % % \end{tabularx}}

    % \section*{Academic Details}
    % \textbf{\emph{Manipal Institute of Technology}}
    % \newline
    % \textit{2018 – 2022 (ongoing)}
    % \newline
    % \textit{B. Tech in Computer Science and Engineering}
    % \newline
    % \textit{CGPA: 7.93}
    % \newline
    % \textit{Number of completed semesters: 3}
    % \newline
    % \textbf{\emph{Amity International School}}
    % \newline
    % \textit{12th CBSE Percentage: 91\% (2015 - 2017)}
    % \newline
    % \textbf{\emph{Suncity World School}}
    % \newline
    % \textit{10th CBSE CGPA: 9.8 (2013 - 2015)}

    % \section*{Interests}
    % Artificial Intelligence,
    % Machine Learning,
    % Deep Learning,
    % Reinforcement Learning,
    % Robot Operating System

%%%%%%%%%%%%%%%%%%%%%%%%%%%%%%%%%%%%%%%%%%%%%%%%%%%%%%%%%%%%%%%%%%%%%%%%%%%%%%
\section*{Internships}
\begin{itemize}
    \item{\textbf{\large{Microsoft-Machine Learning Internship}}
    \newline
    \textit{- Started Learning data structures using Python}
    \newline
    \textit{- Implemented Simple Machine Learning Algorithms using Sklearn,Keras and Tensorflow}}
\end{itemize}
\begin{itemize}
    \item{\textbf{\large{United Nations TakenMind-Global Data Analytics Internship}}
    \newline
    \textit{- Learning Linear and Logistic Regression and implementing it in Python}}
\end{itemize}



% \pagebreak

\section*{Academic Projects}
\begin{itemize}
    \item{\textbf{Food Labs Robotics Startup Interview - ROS Engineer Role}
    \newline
    \textit{- Designed, modelled, constructed and
    Assembled a plethora of sensors and Robots across multiple software platforms like
    freeCad, Blender, Gazebo and also fabricated a hotel from floorplan using Gazebo World Editor}
    \newline
    \textit{- Created an SDF model of the Velodyne HDL-32 sensor,improved the model's appearance and data output,added Mass/Inertia to the model,used freeCad software to acquire Meshes, Blendr software to refine the metric system and Gazebo model editor to model the Velodyne Lidar structure.}
    \newline
    \textit{- Implemented Hokuyo Fake Laser Scanner and Noisy Camera in Gazebo, tweaked the mean and standard deviation of the Gaussian Noise Distribution in the scan and image samples for higher fidelity outputs.}
    \newline
    \textit{- Simulated the ROBOTIS waffle-pi or burger TurtleBot3 and constructed a vehicle in Gazebo using Model editor and loaded it with a Depth Camera Sensor for survelliance}
    \newline
    \textit{- Implemented existing functionality of Clearpath Husky Robot which uses ROS TF for mapping, Adaptive Monte-Carlo Localisation package based on a prticle filter for state estimation and subsequent Localisation and A*/Dijkstra for global plaaning and Dynamic Window approach/TEB local planner for local planning to publish velocity commands to base-controller}}
    \newline
\end{itemize}

\section*{Workshops Attended}
\begin{itemize}
    \item{\textit{Attended Introductory Python Workshop conducted by IE-E\&C
    and got 3 days of hands-on practical Python programming at
    the workshop.}}
    \item{\textit{Attended several Competitive Programming Workshops
    conducted by IECSE and implemented several data structures
    and algorithms in C++ with and without the use of STL Library.}}
    \item{\textit{Attended the Cloud Computing Workshop held by DSC
    Manipal in collaboration with Google Developers Student Club
    where we used Google’s Qwik Labs to implement Machine
    Learning algorithms, Natural Language Processing Algorithms,
    Speech Recognition.}}
    \item{\textit{Attended Machine Learning and Deep Learning Workshops
    given by DSC Manipal and implemented several machine
    learning algorithms using Keras and Tensorflow. }}
    \item{\textit{Attended a 2-day Machine Learning and Deep Learning Workshop 
    conducted by IIT Kharagpur and implemented some basic artificial Neural Networks,
    Convolutional neural networks and Recurrent Neural Networks in Python }}
    \item{\textit{Attended a 6-day Robot Operating System Bootcamp and
    learnt how to use ROS framework to interface Robotics
    components and later used Gazebo and Rviz to simulate
    artificial/real robots in a virtual environment.}}
    \item{\textit{Attended a 3-day Web Development Workshop and implemented 
    the tutorials in HTML, CSS,JavaScript and several of its libraries}}
\end{itemize}

\pagebreak
\section*{Courses Taken}
    \subsection*{College Curriculum}
    Data Structures,
    Object Oriented Programming with Java,
    Digital System Design with Verilog,
    Computer Organization and Architecture,
    Engineering Mathematics. 
    \subsection*{Off-Campus Academies and Online Courses}
    \textbf{Coding Ninjas}-Did C++ programming along with Data Structures 
    and won Top Performer Certificate of Excellence in C++
    \newline
    \textbf{NPTEL}-Basic Electronics,Computer Organization and Architecture, 
    Switching Circuits and Logic Design, Object Oriented Programming with Java,
    
\section*{Positions of Resposibility}
    \begin{itemize}
        \item Local Committee Member of IOSD
        (International Organization of Software Developers)
    \end{itemize}


\section*{Technical Section}
\textbf{Software Familiarity: }
\newline
Anaconda,AutoCAD,Matlab,Keil,Altera MaxPlus 2,VirtualBox,Vm Ware,Oracle SQL,VS Code and Sublime Text 
\newline
\textbf{Programming Languages: }
\newline
Fluent in C/C++,Familiar with Java and Python,Verilog,LaTeX,Linux Shell Scripting,
fair acquaintance with ARM assembly programming\textit{(NXP LPC 1768)}
\newline
\textbf{Libraries and Frameworks: }
\newline
\textbf{Python}-Numpy, Pandas, SciPy,Scikit-Learn,Matplotlib, Keras, Tensorflow    
\newline
\textbf{C++}-Standard Template Library(STL)    
\newline
\textbf{Java}-JavaFX GUI    
\newline
\textbf{Web Developement Languages,Libraries and Frameworks: }
\newline
HTML, CSS, JavaScript and familiarity with MERN stack 
\newline
\textbf{Operating Systems Used: }
\newline
\textbf{Windows}-XP,Vista,7,10
\newline 
\textbf{Linux}-Ubuntu


\section*{Early Years}
\begin{itemize}
    \item Pulled an all nigther in 7th grade to construct a LEGO Mindstorms ev2 
    Humanoid Robot and programmed it using the NXT-G GUI interface that could walk,talk 
    and identify colors in its environment and showcased it it in a Science exhibition.
    \item Won 2nd Prize Trophy at state Level Abacus Competition in 4th grade     
    \item Won 1st Prize Trophy at state Level Abacus Competition in 5th grade
    \item Won some gold ,silver and bronze medals at the \textbf{School Level}  Science 
    and Maths Olypaids conducted by \textbf{Science Olympiad Foundation}            
    
\end{itemize}

% \section*{Personal Details}
% Phone: +91-8800441954
% \newline
% Email: teetangh@gmail.com
% \newline
% GitHub: https://github.com/teetangh

% \renewcommand{\@seccntformat}[1]{}
\end{document}
