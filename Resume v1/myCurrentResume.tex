\documentclass[a4paper,12pt]{article}
\usepackage[utf8]{inputenc}
\usepackage{titlesec}
\usepackage[margin=0.7in]{geometry}
\usepackage{multicol}
\usepackage{tabularx}
\usepackage{hyperref}

\addtolength{\topmargin}{-0.05in}
\begin{document}
    

% \setlength{\columnseprule}{0.4pt}

\begin{multicols}{2}
\section*{\LARGE{Kaustav Ghosh}}
    \section*{Contact Details:}
        Phone: +91-8800441954
        \newline
        Email: \textit{teetangh@gmail.com}
        \newline
        GitHub: \textit{https://github.com/teetangh}
        \newline
        
    \section*{Academic Description}
        \textbf{\emph{Manipal Institute of Technology}}
        \newline
        \textit{2018 – 2022 (ongoing)}
        \newline
        \textit{B. Tech in Computer Science and Engineering}
        % \newline
        % \textit{CGPA: 7.93}
        \newline
        \textit{Number of completed semesters: 4}
        \newline
        \textbf{\emph{Amity International School}}
        \newline
        \textit{12th CBSE Percentage: 91\% (2015 - 2017)}
        \newline
        \textbf{\emph{Suncity World School}}
        \newline
        \textit{10th CBSE CGPA: 9.8 (2013 - 2015)}
\end{multicols}

\section*{Interests}
Artificial Intelligence,
Machine Learning,
Deep Learning,
Reinforcement Learning,
Computer Vision,
Robot Operating System
    
%%%%%%%%%%%%%%%%%%%%%%%%%%%%%%%%%%%%%%%%%%%%%%%%%%%%%%%%%%%%%%%%%%%%%%%%%%%%%%
    % \section*{\LARGE{Kaustav Ghosh}}
    % Phone: +91-8800441954
    % \newline
    % Email: \textit{teetangh@gmail.com}
    % \newline
    % GitHub: \textit{https://github.com/teetangh}

    % % {\begin{tabularx}{\dimexpr\linewidth-25pt}[t]{@{}XXX@{}}
    % % \raggedright
    % % \section*{\Large{Kaustav Ghosh}}
    % %             Phone: +91-8800441954
    % %             \newline
    % %             Email: teetangh@gmail.com
    % %             \newline
    % %             GitHub: https://github.com/teetangh
        
    % %         & 
    % % % \centering  & 
    % % \raggedleft 
    % % \end{tabularx}}

    % \section*{Academic Details}
    % \textbf{\emph{Manipal Institute of Technology}}
    % \newline
    % \textit{2018 – 2022 (ongoing)}
    % \newline
    % \textit{B. Tech in Computer Science and Engineering}
    % \newline
    % \textit{CGPA: 7.93}
    % \newline
    % \textit{Number of completed semesters: 3}
    % \newline
    % \textbf{\emph{Amity International School}}
    % \newline
    % \textit{12th CBSE Percentage: 91\% (2015 - 2017)}
    % \newline
    % \textbf{\emph{Suncity World School}}
    % \newline
    % \textit{10th CBSE CGPA: 9.8 (2013 - 2015)}

    % \section*{Interests}
    % Artificial Intelligence,
    % Machine Learning,
    % Deep Learning,
    % Reinforcement Learning,
    % Robot Operating System

%%%%%%%%%%%%%%%%%%%%%%%%%%%%%%%%%%%%%%%%%%%%%%%%%%%%%%%%%%%%%%%%%%%%%%%%%%%%%%

\section*{Internships}
\begin{itemize}
    \item{\textbf{\large{Microsoft Student Partners-Machine Learning Internship}}
    \newline
    \textbf{ ML Certificate :} \href{https://github.com/teetangh/Kaustav-All-Certifications/blob/master/Artificial%20Intelligence/MSP%20ML%20Internship/internship%20certificate.jpg}{Certificate1}.
    \textbf{ Python Foundations Certificate :} \href{https://github.com/teetangh/Kaustav-All-Certifications/blob/master/Artificial%20Intelligence/MSP%20ML%20Internship/Course%20Completion%20Certificate.jpg}{Certificate2}.
    \textbf{ Team Repository :} \href{https://github.com/Microsoft-ML-Internship-Team/Major-Project-Submissions}{Repository}.
    \newline
    \textit{- Guided a team of 10 individuals to collaborate and accomplish a Regression task of price prediction of used cars}
    \newline
    \textit{- Performed Feature Engineering to detect the most important attributes of the dataset using Univariate and Multivarte Filtering techinuqes, Mutual Entropy Gain Filtering and also feature selection using RMSE Regression and ANOVA Test  }
    \newline
    \textit{- Performed basic Data wrangling and precessing using numpy and pandas and visualized it using matplotlib and seaborn and finally buit the machine learning model using an XGboost Regressor }
    \newline
    \textit{- Also completed a Mini Project on extensive Data Visualization and Analysis using Matplotlib and Seaborn to gather useful insights of the data}
    \newline
    \textbf{ Mini Project:} \href{https://github.com/teetangh/Microsoft-Machine-Learning-Internship/blob/master/MINOR%20PROJECT/Microsoft_Minor_Project_v2.ipynb}{Notebook}.
    \newline
    \textbf{ Feature Engineering:} \href{https://github.com/Microsoft-ML-Internship-Team/Major-Project-Submissions/blob/master/KAUSTAV/02_Kaustav_feature_engineering_v4.ipynb}{Notebook}.
    \textbf{ Model:} \href{https://github.com/Microsoft-ML-Internship-Team/Major-Project-Submissions/blob/master/KAUSTAV/03_Kaustav_Buidling_the_model_v1.ipynb}{Notebook}.
    \textbf{ EDA:} \href{https://github.com/Microsoft-ML-Internship-Team/Major-Project-Submissions/blob/master/KAUSTAV/01_Kaustav_data_preprocess_EDA_v7.ipynb}{Notebook}.}
\end{itemize}
\begin{itemize}
    \item{\textbf{\large{United Nations TakenMind-Global Data Analytics Internship}}
    \newline
    \textbf{Letter of recommendation :} \href{https://github.com/teetangh/Kaustav-All-Certifications/blob/master/Artificial%20Intelligence/UN%20Takenmind%20Data%20Analytics%20Internship/35799-KaustavGhosh-recommendation-letter.pdf}{LOR}.
    \textbf{Completion Certificate :} \href{https://github.com/teetangh/Kaustav-All-Certifications/blob/master/Artificial%20Intelligence/UN%20Takenmind%20Data%20Analytics%20Internship/35799-KaustavGhosh.pdf}{Certificate}.
    \newline
    \textit{- Scripted a personal version of Numpy and Pandas Documentation }
    \newline
    \textit{- Performed Exploratory Data Analyis techniques using Matplotlib and Seaborn }
    \newline
    \textit{- Creating several boxplots,countplots,heatmaps of several datasets}}
\end{itemize}
\begin{itemize}
    \item{\textbf{\large{Ineuron Deep Learning with Computer Vision and Natural Language Processing Internship}}
    \newline
    \textit{- Currently learning CNNs and RNNs }}
\end{itemize}



\pagebreak

\section*{Academic Projects}
\begin{itemize}
    \item{\textbf{\large{Food Labs Robotics Startup Interview - ROS Engineer Role}}.
    \newline
    \textbf{ Models and Simulations :} \href{https://github.com/teetangh/Kaustav-ROS-Workspace}{Repository \& videos}.
    \textbf{  Final Project Report:} \href{https://github.com/teetangh/Kaustav-ROS-Workspace/blob/master/Resources/Final%20Report.pdf}{Final Report}.
    \newline
    \textit{- Designed, modelled, constructed and
    Assembled a plethora of sensors and Robots across multiple software platforms like
    freeCad, Blender, Gazebo and also fabricated a hotel from floorplan using Gazebo World Editor}
    \newline
    \textit{- Created an SDF model of the Velodyne HDL-32 sensor,improved the model's appearance and data output,added Mass/Inertia to the model,used freeCad software to acquire Meshes, Blendr software to refine the metric system and Gazebo model editor to model the Velodyne Lidar structure.}
    \newline
    \textit{- Implemented Hokuyo Fake Laser Scanner and Noisy Camera in Gazebo, tweaked the mean and standard deviation of the Gaussian Noise Distribution in the scan and image samples for higher fidelity outputs.}
    \newline
    \textit{- Simulated the ROBOTIS waffle-pi or burger TurtleBot3 and constructed a vehicle in Gazebo using Model editor and loaded it with a Depth Camera Sensor for survelliance}
    \newline
    \textit{- Implemented existing functionality of Clearpath Husky Robot which uses ROS TF for mapping, Adaptive Monte-Carlo Localisation package based on a prticle filter for state estimation and subsequent Localisation and A*/Dijkstra for global planning and Dynamic Window approach/Timed Elastic Band local planner for local planning to publish velocity commands to base-controller}
    \newline
    \textit{\textbf{Sidenote:} Original Project was Forked by Blender 3D computer graphics software }}
    \newline

    \item{\textbf{\large{Analysis of Selective Compliance Assembly Robot Arm and Modelling of T3R Robot}}
    \newline
    \textit{- Computed Denavit–Hartenberg parameters for the SCARA robot and used it to formulate the Forward and Inverse Kinematics of the robot arm} 
    \newline
    \textit{- Used Lagrange Euler Formulation to compute the torque/dynamics of the robot and further also planned an arbitrary trajectory for the manipulator} 
    \newline
    \textit{- Using Solidworks modelled a T3R robot (1 twisting joint and 3 revolute joints) and as bonus task i am trying to interface the Soliworks model with Matlab Simscape}} 
    \newline
    \textbf{ Link to SCARA :} \href{https://github.com/teetangh/Robotics-Projects/blob/master/SCARA_Robot_Analysis.pdf}{SCARA Analysis}.
    \newline
    \textbf{ Link to T3R :} \href{https://github.com/teetangh/Robotics-Projects/tree/master/T3R%20Robot}{Solidworks Model}.
    \textbf{ Link to T3R animation :} \href{https://github.com/teetangh/Robotics-Projects/blob/master/T3R%20Robot/Resources/T3R%20Animation.mp4}{mp4 video}.
    \newline
    \item{\textbf{\large{Finland Labs in association with NSS IIT Roorkee - Covid 19 Data Analysis, Time Series Forecasting and Web Scraping}}
    % Don't know why indentation is not working
    \newline
    \textbf{Link to Certificate :} \href{https://github.com/teetangh/Kaustav-All-Certifications/blob/master/Artificial%20Intelligence/Finland%20Labs%20and%20IITR/Covid%2019%20Analysis%20-%20AI%20and%20ML.pdf}{Certificate}.
    \newline
    \textbf{ Link to Code :} \href{https://github.com/teetangh/FinlandLabs-IITR-COVID-19-Analysis}{Repository}.
    \newline
    \textit{- Prepared a complete Data Analyis report on the World-wide COVID-19 attack statistics and used the Facebook's fbprophet Time-series Forecasting library to speculate the number of active corona victim cases in the upcoming days.} 
    \newline
    \textit{- Also used a corona dataset of my country and the Python folium package for the binding of data to a map for choropleth visualizations. Further used beautifulSoup and Requests HTTP library for Web Scraping of live corona stats.}
    \newline
    \textit{- Implemented code snippets for the pre-processing of data, data wrangling and visualized the data via several Matplotlib and Seaborn tools }
    \newline
    \textit{- Created neural networks from scratch which facilitated in implementing a machine learning model to recognize the function of an XOR gate without explicitily being programmed.}
    \newline
    \textit{- Trained a Deep Learning model using Tensorflow and Keras API for MNIST Handwritten digit Recognition}}
    
\end{itemize}

\section*{Workshops Attended}
    \textbf{Link to Repository :} \href{https://github.com/teetangh/Attended-Workshops}{Repository}.
\begin{itemize}
    \item{\textit{Attended Introductory Python Workshop conducted by IE-E\&C
    and got 3 days of hands-on practical Python programming at
    the workshop.}}
    \item{\textit{Attended several Competitive Programming Workshops
    conducted by IECSE and implemented several data structures
    and algorithms in C++ with and without the use of STL Library.}}
    \item{\textit{Attended Image Processing and Computer Vision Workshop by ISTE and implemented basic OpenCV programs }}
    \textbf{Link to Certificate :} \href{https://github.com/teetangh/Kaustav-All-Certifications/blob/master/Computer%20Vision%20and%20NLP/ISTE%20Image%20Processing%20using%20OpenCV/180905188.jpg}{Certificate}.
    \item{\textit{Attended the Cloud Computing Workshop held by DSC
    Manipal in collaboration with Google Developers Student Club
    where we used Google’s Qwik Labs to implement Machine
    Learning algorithms, Natural Language Processing Algorithms,
    Speech Recognition.}}
    \item{\textit{Attended Machine Learning and Deep Learning Workshops
    given by DSC Manipal and implemented several machine
    learning algorithms using Keras and Tensorflow. }}
    \item{\textit{Attended a 2-day Machine Learning and Deep Learning Workshop 
    conducted by IIT Kharagpur and implemented some basic artificial Neural Networks,
    Convolutional neural networks and Recurrent Neural Networks in Python }}
    \item{\textit{Attended a 6-day Robot Operating System Bootcamp and
    learnt how to use ROS framework to interface Robotics
    components and later used Gazebo and Rviz to simulate
    artificial/real robots in a virtual environment.}}
    \item{\textit{Attended a 3-day Web Development Workshop and implemented 
    the tutorials in HTML, CSS,JavaScript and several of its libraries}}
    \textbf{Link to Certificate :} \href{https://github.com/teetangh/Kaustav-All-Certifications/blob/master/Web%20Development/IECSE%20-%20Web.IO%20Certificate.pdf}{Certificate}
\end{itemize}

% \pagebreak
\section*{Courses Taken}
    \subsection*{College Curriculum}
    Engineering Mathematics, 
    Data Structures,
    Object Oriented Programming with Java,
    Digital System Design with Verilog,
    Computer Organization and Architecture,
    Database Systems,
    Theory of Computation,
    Embedded Systems,
    Algorithms,
    Robotics 
    \subsection*{Off-Campus Academies and Online Courses}
    \textbf{Coding Ninjas}-Did C++ programming along with Data Structures 
    and won Top Performer Certificate of Excellence in C++
    \newline
    \textbf{Link to Completion Certificate :} \href{https://github.com/teetangh/Kaustav-All-Certifications/blob/master/Programming/Coding%20Ninjas/Coding%20Ninjas%20Cpp%20Completion%20Certificate.pdf}{Certificate}
    \newline
    \textbf{Link to Top Performer Certificate :} \href{https://github.com/teetangh/Kaustav-All-Certifications/blob/master/Programming/Coding%20Ninjas/Coding%20Ninjas%20Cpp%20Top%20Performer%20Certificate.pdf}{Certificate}
    \newline
    \textbf{Link to Cpp and Data strucutres Repository :} \href{https://github.com/teetangh/Kaustav-Competitive-Coding}{Repository}
    \newline
    \textbf{NPTEL}-Basic Electronics,Switching Circuits \& Logic Design,Computer Organization \& Architecture,Object Oriented Programming with Java
    
\section*{Positions of Resposibility}
    \begin{itemize}
        \item Local Committee Member of IOSD
        (International Organization of Software Developers)
    \end{itemize}


\section*{Technical Section}
\textbf{Software Familiarity: }
\newline
Anaconda,AutoCAD,Matlab,Keil,Altera MaxPlus 2,VirtualBox,Vm Ware,Oracle SQL,VS Code and Sublime Text 
\newline
\textbf{Programming Languages: }
\newline
Fluent in C/C++,Familiar with Java and Python,Verilog,LaTeX,Linux Shell Scripting,
fair acquaintance with ARM assembly programming\textit{(NXP LPC 1768)}
\newline
\textbf{Libraries and Frameworks: }
\newline
\textbf{Python}-Numpy, Pandas, SciPy,Scikit-Learn,Matplotlib, Keras, Tensorflow    
\newline
\textbf{C++}-Standard Template Library(STL)    
\newline
\textbf{Java}-JavaFX GUI    
\newline
\textbf{Robotics Libraries and Frameworks: }
\newline
ROS middleware, Gazebo, Ignition, MoveIt!, Point Cloud Library
\newline
\textbf{Web Developement Languages,Libraries and Frameworks: }
\newline
HTML, CSS, JavaScript and familiarity with MERN stack
\newline
\textbf{Operating Systems Used: }
\newline
\textbf{Windows}-XP,Vista,7,10 
\textbf{Linux}-Ubuntu

\section*{Early Years}
\begin{itemize}
    \item Pulled an all nigther in 7th grade to construct a LEGO Mindstorms ev2 
    Humanoid Robot and programmed it using the NXT-G GUI interface that could walk,talk 
    and identify colors in its environment and showcased it it in a Science exhibition.
    \item Won Gold medal for 200m sprint and 2 bronze medals fro 100m and 400m in 8th grade.
    \item Won 1st Prize Trophy at state Level Abacus Competition in 5th grade
    \item Won 2nd Prize Trophy at state Level Abacus Competition in 4th grade     
    \item Won some gold, silver and bronze medals at the \textbf{School Level}  Science 
    and Maths Olypaids conducted by \textbf{Science Olympiad Foundation}            
\end{itemize}

% \section*{Personal Details}
% Phone: +91-8800441954
% \newline
% Email: teetangh@gmail.com
% \newline
% GitHub: https://github.com/teetangh

% \renewcommand{\@seccntformat}[1]{}
\end{document}
